\subsection*{Alcance}

El proyecto pilar\+Trimino\+Maven\+Jar parte del proyecto realizado en la asignatura de Programacion II de Ingeniería Técnica Informática cursada en la U\+N\+ED. En la asignatura de Calidad de Software del Grado en Ingeniería Informática(E\+H\+U/\+U\+PV) desarrollamos herramientas para documentación como Maven. Estudio esta herramienta aplicándola al proyecto desarrollado en Java.

El proyecto estaba dirigido a la eficiencia en el uso de algoritmos a la hora de plantear cómo realizar un trimino dadas una dimensión de matriz y una posición (x,y) de comienzo.

\subsection*{Mejoras realizadas}

1-\/ Documentación del proyecto con Maven\+: Creo un proyecto Maven y le añado las clases que tenía en el proyecto del que parto. Esto me permite conocer desde la experiencia la estructura de los proyectos Maven.

2-\/ Como consecuencia veo la necesidad de realizar algún test puesto que en Java no tengo gran experiencia y no había realizado test previamente.

3-\/ Análisis de calidad de estilo\+:
\begin{DoxyItemize}
\item En la asigatura de Calidad de Sotware descubro la herramienta para eclipse checkstyle, que me permite encontrar todos las advertencias y posibles errores de estilo del código. La implemento en mi proyecto obtiendo numerosos warnings.
\item Descubro la posibilidad de que el propio eclipse elimine los warnings casi en su totalidad con source/\+Format
\end{DoxyItemize}

\subsection*{Posibles mejoras futuras}


\begin{DoxyItemize}
\item Aplicar Doxigen
\item Hacer más test y seguir investigando las posibilidades de Maven
\end{DoxyItemize}

\subsection*{Enlace a Git\+Hub del proyecto}

\href{https://github.com/cs-ehu/pilarTriminoMavenJar}{\texttt{ https\+://github.\+com/cs-\/ehu/pilar\+Trimino\+Maven\+Jar}} 